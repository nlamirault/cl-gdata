
    \documentclass[a4paper]{report}
    \pagestyle{plain}
    %\usepackage{amsmath}
    %\usepackage{amssymb}
    %\usepackage{amsthm}
    %\usepackage{fancyvrb}
    \usepackage[usenames]{color}
    \usepackage[linkcolor=black,citecolor=black]{hyperref}
    \usepackage{makeidx}
    \makeindex

    \setlength{\parindent}{0cm}
    \addtolength{\parskip}{0.5em}
    %\renewcommand{\thefootnote}{\fnsymbol{footnote}}

    \begin{document}
    \input{defun.tex}

    \title{cl-gdata API reference}
    %\date{}
    \maketitle

    \tableofcontents
    \newpage

    
    \chapter{The cl-gdata-service package}
    cl-gdata-service package.
      \section{Other functions}
      

    \rule{\linewidth}{0.1mm}
    
    \label{cl-gdata-service__fun__authenticates}
    \begin{defun}[Function]
    authenticates gdata-service


    
    \bigskip
    \textsc{Arguments}

gdata-service
	--- a \hyperref[cl-gdata-service__class__gdata-service]{\texttt{gdata-service}}
   instance




      
    \bigskip
    \textsc{Condition Types Signalled}


 	
    \begin{itemize}
    
	  
    \item
    \hyperref[cl-gdata-service__class__gdata-service-error]{gdata-service-error}
    
    \item
    \hyperref[cl-gdata-service__class__gdata-error]{gdata-error}
    
	
    \end{itemize}
  
      


	
    \bigskip
    \textsc{Details}

Make a HTTP request to authenticate.
If HTTP return code is 200, update the AUTH token of gdata-service.


      
    \bigskip
    \textsc{See also}


	
    \begin{itemize}
    
	  
    \item
    \hyperref[cl-gdata-service__class__gdata-service]{gdata-service}
    
	
    \end{itemize}
  
      


    
    \end{defun}
  
  

    \rule{\linewidth}{0.1mm}
    
    \label{cl-gdata-service__fun__check-account-type}
    \begin{defun}[Function]
    check-account-type account-type


    
    \bigskip
    \textsc{Arguments}

account-type
	--- The Google account type. Must be one of\texttt{:google}, \texttt{:hosted-or-google} or \texttt{:hosted}




    
    \bigskip
    \textsc{Return Values}

True is this Google account type is available.


      
    \bigskip
    \textsc{Condition Types Signalled}


 	
    \begin{itemize}
    
	  
    \item
    \hyperref[cl-gdata-service__class__gdata-error]{gdata-error}
    
	
    \end{itemize}
  
      


	
    \bigskip
    \textsc{Details}

Check if the Google account type is available.


    
    \end{defun}
  
  

    \rule{\linewidth}{0.1mm}
    
    \label{cl-gdata-service__fun__gdata-service-account-type}
    \begin{defun}[Function]
    gdata-service-account-type object


    
    \bigskip
    \textsc{Arguments}

instance
	--- a \hyperref[cl-gdata-service__class__gdata-service]{\texttt{gdata-service}}
  




    
    \bigskip
    \textsc{Return Values}

a keyword


	
    \bigskip
    \textsc{Details}

              Return the type of account to use. Use \texttt{:google} forregular Google accounts or \texttt{:hosted} for Google Apps accounts,or \texttt{:hosted-or-google} to try finding a hosted account first and,
if it doesn't exist, try finding a regular Google account.Default value: \texttt{:hosted-or-google}.


      
    \bigskip
    \textsc{See also}


	
    \begin{itemize}
    
	  
    \item
    \hyperref[cl-gdata-service__class__gdata-service]{gdata-service}
    
	
    \end{itemize}
  
      


    
    \end{defun}
  
  

    \rule{\linewidth}{0.1mm}
    
    \label{cl-gdata-service__fun__gdata-service-auth}
    \begin{defun}[Function]
    gdata-service-auth object


    
    \bigskip
    \textsc{Arguments}

instance
	--- a \hyperref[cl-gdata-service__class__gdata-service]{\texttt{gdata-service}}
  




    
    \bigskip
    \textsc{Return Values}

a string


	
    \bigskip
    \textsc{Details}

              The auth token used for authenticating requests


      
    \bigskip
    \textsc{See also}


	
    \begin{itemize}
    
	  
    \item
    \hyperref[cl-gdata-service__class__gdata-service]{gdata-service}
    
	
    \end{itemize}
  
      


    
    \end{defun}
  
  

    \rule{\linewidth}{0.1mm}
    
    \label{cl-gdata-service__fun__gdata-service-email}
    \begin{defun}[Function]
    gdata-service-email object


    
    \bigskip
    \textsc{Arguments}

instance
	--- a \hyperref[cl-gdata-service__class__gdata-service]{\texttt{gdata-service}}
  




    
    \bigskip
    \textsc{Return Values}

a string


	
    \bigskip
    \textsc{Details}

              Return the email of the Google account.


      
    \bigskip
    \textsc{See also}


	
    \begin{itemize}
    
	  
    \item
    \hyperref[cl-gdata-service__class__gdata-service]{gdata-service}
    
	
    \end{itemize}
  
      


    
    \end{defun}
  
  

    \rule{\linewidth}{0.1mm}
    
    \label{cl-gdata-service__fun__gdata-service-password}
    \begin{defun}[Function]
    gdata-service-password object


    
    \bigskip
    \textsc{Arguments}

instance
	--- a \hyperref[cl-gdata-service__class__gdata-service]{\texttt{gdata-service}}
  




    
    \bigskip
    \textsc{Return Values}

a string


	
    \bigskip
    \textsc{Details}

              Return the password of the Google account.


      
    \bigskip
    \textsc{See also}


	
    \begin{itemize}
    
	  
    \item
    \hyperref[cl-gdata-service__class__gdata-service]{gdata-service}
    
	
    \end{itemize}
  
      


    
    \end{defun}
  
  

    \rule{\linewidth}{0.1mm}
    
    \label{cl-gdata-service__fun__gdata-service-service}
    \begin{defun}[Function]
    gdata-service-service object


    
    \bigskip
    \textsc{Arguments}

instance
	--- a \hyperref[cl-gdata-service__class__gdata-service]{\texttt{gdata-service}}
  




    
    \bigskip
    \textsc{Return Values}

a string


	
    \bigskip
    \textsc{Details}

       
       The desired service for which credentials willbe obtained.


      
    \bigskip
    \textsc{See also}


	
    \begin{itemize}
    
	  
    \item
    \hyperref[cl-gdata-service__class__gdata-service]{gdata-service}
    
	
    \end{itemize}
  
      


    
    \end{defun}
  
  

    \rule{\linewidth}{0.1mm}
    
    \label{cl-gdata-service__fun__gdata-service-source}
    \begin{defun}[Function]
    gdata-service-source object


    
    \bigskip
    \textsc{Arguments}

instance
	--- a \hyperref[cl-gdata-service__class__gdata-service]{\texttt{gdata-service}}
  




    
    \bigskip
    \textsc{Return Values}

a string


	
    \bigskip
    \textsc{Details}

              Return the name of the user's application.


      
    \bigskip
    \textsc{See also}


	
    \begin{itemize}
    
	  
    \item
    \hyperref[cl-gdata-service__class__gdata-service]{gdata-service}
    
	
    \end{itemize}
  
      


    
    \end{defun}
  
  
      \section{Other classes}
      

    \rule{\linewidth}{0.1mm}
    
    \label{cl-gdata-service__class__gdata-error}
    \begin{defun}[Class]
    gdata-error


      
    \bigskip
    \textsc{Superclasses}

\color[rgb]{0.5,0.5,0.5}common-lisp:simple-error\color[rgb]{0,0,0}, \color[rgb]{0.5,0.5,0.5}common-lisp:simple-condition\color[rgb]{0,0,0}, \color[rgb]{0.5,0.5,0.5}common-lisp:error\color[rgb]{0,0,0}, \color[rgb]{0.5,0.5,0.5}common-lisp:serious-condition\color[rgb]{0,0,0}, \color[rgb]{0.5,0.5,0.5}common-lisp:condition\color[rgb]{0,0,0}, \color[rgb]{0.5,0.5,0.5}sb-pcl::slot-object\color[rgb]{0,0,0}, \color[rgb]{0.5,0.5,0.5}common-lisp:t\color[rgb]{0,0,0}


      
    \bigskip
    \textsc{Documented Subclasses}

\hyperref[cl-gdata-service__class__gdata-request-error]{
	  gdata-request-error
	}
      , \hyperref[cl-gdata-service__class__gdata-service-error]{
	  gdata-service-error
	}
      


	
    \bigskip
    \textsc{Details}

GData main error.


    
    \end{defun}
  
  

    \rule{\linewidth}{0.1mm}
    
    \label{cl-gdata-service__class__gdata-request-error}
    \begin{defun}[Class]
    gdata-request-error


      
    \bigskip
    \textsc{Superclasses}

\hyperref[cl-gdata-service__class__gdata-error]{
	  gdata-error
	}
      , \color[rgb]{0.5,0.5,0.5}common-lisp:simple-error\color[rgb]{0,0,0}, \color[rgb]{0.5,0.5,0.5}common-lisp:simple-condition\color[rgb]{0,0,0}, \color[rgb]{0.5,0.5,0.5}common-lisp:error\color[rgb]{0,0,0}, \color[rgb]{0.5,0.5,0.5}common-lisp:serious-condition\color[rgb]{0,0,0}, \color[rgb]{0.5,0.5,0.5}common-lisp:condition\color[rgb]{0,0,0}, \color[rgb]{0.5,0.5,0.5}sb-pcl::slot-object\color[rgb]{0,0,0}, \color[rgb]{0.5,0.5,0.5}common-lisp:t\color[rgb]{0,0,0}


      
    \bigskip
    \textsc{Documented Subclasses}


	    None
	  


	
    \bigskip
    \textsc{Details}

Condition raised when an invalide request to theGoogle web services is performed.


    
    \end{defun}
  
  

    \rule{\linewidth}{0.1mm}
    
    \label{cl-gdata-service__class__gdata-service}
    \begin{defun}[Class]
    gdata-service


      
    \bigskip
    \textsc{Superclasses}

\color[rgb]{0.5,0.5,0.5}common-lisp:standard-object\color[rgb]{0,0,0}, \color[rgb]{0.5,0.5,0.5}sb-pcl::slot-object\color[rgb]{0,0,0}, \color[rgb]{0.5,0.5,0.5}common-lisp:t\color[rgb]{0,0,0}


      
    \bigskip
    \textsc{Documented Subclasses}

\hyperref[cl-gdata-picasa__class__gdata-picasa]{
	  gdata-picasa
	}
      


      
    \bigskip
    \textsc{Slot Access Functions}


	
    \begin{itemize}
    
	  
    \item
    \hyperref[cl-gdata-service__fun__gdata-service-email]{gdata-service-email}
    
    \item
    \hyperref[cl-gdata-service__fun__gdata-service-password]{gdata-service-password}
    
    \item
    \hyperref[cl-gdata-service__fun__gdata-service-account-type]{gdata-service-account-type}
    
    \item
    \hyperref[cl-gdata-service__fun__gdata-service-service]{gdata-service-service}
    
    \item
    \hyperref[cl-gdata-service__fun__gdata-service-source]{gdata-service-source}
    
    \item
    \hyperref[cl-gdata-service__fun__gdata-service-auth]{gdata-service-auth}
    
	
    \end{itemize}
  
      


	
    \bigskip
    \textsc{Details}

The superclass of all Google service.Subclasses of \texttt{gdata-service} represents Google service available using
cl-gdata : Picasa.


    
    \end{defun}
  
  

    \rule{\linewidth}{0.1mm}
    
    \label{cl-gdata-service__class__gdata-service-error}
    \begin{defun}[Class]
    gdata-service-error


      
    \bigskip
    \textsc{Superclasses}

\hyperref[cl-gdata-service__class__gdata-error]{
	  gdata-error
	}
      , \color[rgb]{0.5,0.5,0.5}common-lisp:simple-error\color[rgb]{0,0,0}, \color[rgb]{0.5,0.5,0.5}common-lisp:simple-condition\color[rgb]{0,0,0}, \color[rgb]{0.5,0.5,0.5}common-lisp:error\color[rgb]{0,0,0}, \color[rgb]{0.5,0.5,0.5}common-lisp:serious-condition\color[rgb]{0,0,0}, \color[rgb]{0.5,0.5,0.5}common-lisp:condition\color[rgb]{0,0,0}, \color[rgb]{0.5,0.5,0.5}sb-pcl::slot-object\color[rgb]{0,0,0}, \color[rgb]{0.5,0.5,0.5}common-lisp:t\color[rgb]{0,0,0}


      
    \bigskip
    \textsc{Documented Subclasses}


	    None
	  


	
    \bigskip
    \textsc{Details}

Condition raised when the Google web services are down.


    
    \end{defun}
  
  
      \section{Other variables}
      

    \rule{\linewidth}{0.1mm}
    
    \label{cl-gdata-service__variable__+gdata-atom+}
    \begin{defun}[Variable]
    +gdata-atom+


	No documentation string.  Possibly unimplemented or incomplete.
	


    
    \end{defun}
  
  

    \rule{\linewidth}{0.1mm}
    
    \label{cl-gdata-service__variable__+gdata-bad-request+}
    \begin{defun}[Variable]
    +gdata-bad-request+


	
    \bigskip
    \textsc{Details}

Invalid request URI or header, or unsupported nonstandard parameter.


    
    \end{defun}
  
  

    \rule{\linewidth}{0.1mm}
    
    \label{cl-gdata-service__variable__+gdata-conflict+}
    \begin{defun}[Variable]
    +gdata-conflict+


	
    \bigskip
    \textsc{Details}

Specified version number doesn't match resource's latest version number.


    
    \end{defun}
  
  

    \rule{\linewidth}{0.1mm}
    
    \label{cl-gdata-service__variable__+gdata-created+}
    \begin{defun}[Variable]
    +gdata-created+


	
    \bigskip
    \textsc{Details}

Creation of a resource was successful.


    
    \end{defun}
  
  

    \rule{\linewidth}{0.1mm}
    
    \label{cl-gdata-service__variable__+gdata-forbidden+}
    \begin{defun}[Variable]
    +gdata-forbidden+


	
    \bigskip
    \textsc{Details}

Unsupported standard parameter, or authentication or authorization failed.


    
    \end{defun}
  
  

    \rule{\linewidth}{0.1mm}
    
    \label{cl-gdata-service__variable__+gdata-internal-error+}
    \begin{defun}[Variable]
    +gdata-internal-error+


	
    \bigskip
    \textsc{Details}

Internal error. This is the default code that is used for all unrecognizederrors.


    
    \end{defun}
  
  

    \rule{\linewidth}{0.1mm}
    
    \label{cl-gdata-service__variable__+gdata-not-found+}
    \begin{defun}[Variable]
    +gdata-not-found+


	
    \bigskip
    \textsc{Details}

Resource (such as a feed or entry) not found.


    
    \end{defun}
  
  

    \rule{\linewidth}{0.1mm}
    
    \label{cl-gdata-service__variable__+gdata-not-modified+}
    \begin{defun}[Variable]
    +gdata-not-modified+


	
    \bigskip
    \textsc{Details}

The resource hasn't changed since the time specified in the request'sIf-Modified-Since header.


    
    \end{defun}
  
  

    \rule{\linewidth}{0.1mm}
    
    \label{cl-gdata-service__variable__+gdata-ok+}
    \begin{defun}[Variable]
    +gdata-ok+


	
    \bigskip
    \textsc{Details}

No error.


    
    \end{defun}
  
  

    \rule{\linewidth}{0.1mm}
    
    \label{cl-gdata-service__variable__+gdata-unauthorized+}
    \begin{defun}[Variable]
    +gdata-unauthorized+


	
    \bigskip
    \textsc{Details}

Authorization required.


    
    \end{defun}
  
  

    \rule{\linewidth}{0.1mm}
    
    \label{cl-gdata-service__variable___source_}
    \begin{defun}[Variable]
    *source*


	No documentation string.  Possibly unimplemented or incomplete.
	


    
    \end{defun}
  
  
    \chapter{The cl-gdata-picasa package}
    cl-gdata-picasa package.
The Picasa Web Albums Data API allows for websites and programs to integrate with Picasa Web Albums.
      \section{Other functions}
      

    \rule{\linewidth}{0.1mm}
    
    \label{cl-gdata-picasa__fun__add-album}
    \begin{defun}[Function]
    add-album gdata-picasa user title summary location access keywords


    
    \bigskip
    \textsc{Arguments}

gdata-picasa
	--- a \hyperref[cl-gdata-picasa__class__gdata-picasa]{\texttt{gdata-picasa}}
   instance

user
	--- An username

title
	--- Title of the photos albun

summary
	--- A summary / description for this album

location
	--- Place for a geolocation of this album

access
	--- \texttt{private} or \texttt{public}. Public albums are searchable by everyone on the internet. Defaults to \texttt{public}

keywords
	--- A String, keywords separated by coma




    
    \bigskip
    \textsc{Return Values}

A string


	
    \bigskip
    \textsc{Details}

Add a new album for user. Needs authentication.Return a string which contains XML format representing the new album feed.


      
    \bigskip
    \textsc{See also}


	
    \begin{itemize}
    
	  
    \item
    \hyperref[cl-gdata-picasa__class__gdata-picasa]{gdata-picasa}
    
	
    \end{itemize}
  
      


    
    \end{defun}
  
  

    \rule{\linewidth}{0.1mm}
    
    \label{cl-gdata-picasa__fun__add-photo}
    \begin{defun}[Function]
    add-photo gdata-picasa user album filename


    
    \bigskip
    \textsc{Arguments}

gdata-picasa
	--- a \hyperref[cl-gdata-picasa__class__gdata-picasa]{\texttt{gdata-picasa}}
   instance

user
	--- An username

album
	--- Name of the photos albun

filenanem
	--- The image filename




    
    \bigskip
    \textsc{Return Values}

A string


	
    \bigskip
    \textsc{Details}

Add a photo for user's album. Needs authentication.Return a string which contains XML format representing the photo feed.


      
    \bigskip
    \textsc{See also}


	
    \begin{itemize}
    
	  
    \item
    \hyperref[cl-gdata-picasa__class__gdata-picasa]{gdata-picasa}
    
	
    \end{itemize}
  
      


    
    \end{defun}
  
  

    \rule{\linewidth}{0.1mm}
    
    \label{cl-gdata-picasa__fun__list-albums}
    \begin{defun}[Function]
    list-albums gdata-picasa user &key authentication-p


    
    \bigskip
    \textsc{Arguments}

gdata-picasa
	--- a \hyperref[cl-gdata-picasa__class__gdata-picasa]{\texttt{gdata-picasa}}
   instance

user
	--- An username

authentication-p
	--- Specify if the request is done with authentification or not




    
    \bigskip
    \textsc{Return Values}

A string


      
    \bigskip
    \textsc{Condition Types Signalled}


 	
    \begin{itemize}
    
	  
    \item
    \hyperref[cl-gdata-service__class__gdata-request-error]{gdata-request-error}
    
	
    \end{itemize}
  
      


	
    \bigskip
    \textsc{Details}

Get a feed of an user's albums.Return a string in XML format which contains the list of albums.


      
    \bigskip
    \textsc{See also}


	
    \begin{itemize}
    
	  
    \item
    \hyperref[cl-gdata-picasa__class__gdata-picasa]{gdata-picasa}
    
	
    \end{itemize}
  
      


    
    \end{defun}
  
  

    \rule{\linewidth}{0.1mm}
    
    \label{cl-gdata-picasa__fun__list-photos}
    \begin{defun}[Function]
    list-photos gdata-picasa user album &key authentication-p


    
    \bigskip
    \textsc{Arguments}

gdata-picasa
	--- a \hyperref[cl-gdata-picasa__class__gdata-picasa]{\texttt{gdata-picasa}}
   instance

user
	--- An username

album
	--- Name of the photos albun

authentication-p
	--- Specify if the request is done with authentification or not




    
    \bigskip
    \textsc{Return Values}

A string


      
    \bigskip
    \textsc{Condition Types Signalled}


 	
    \begin{itemize}
    
	  
    \item
    \hyperref[cl-gdata-service__class__gdata-request-error]{gdata-request-error}
    
	
    \end{itemize}
  
      


	
    \bigskip
    \textsc{Details}

Get a feed listing all of the photos in an album belonging
to an user.
Return a string in XML format which contains the list of photos in this album.


      
    \bigskip
    \textsc{See also}


	
    \begin{itemize}
    
	  
    \item
    \hyperref[cl-gdata-picasa__class__gdata-picasa]{gdata-picasa}
    
	
    \end{itemize}
  
      


    
    \end{defun}
  
  

    \rule{\linewidth}{0.1mm}
    
    \label{cl-gdata-picasa__fun__make-gdata-picasa}
    \begin{defun}[Function]
    make-gdata-picasa email password &key (account-type hosted-or-google)


    
    \bigskip
    \textsc{Arguments}

email
	--- The email of the Google account

password
	--- The password of the Google account




      
    \bigskip
    \textsc{Condition Types Signalled}


 	
    \begin{itemize}
    
	  
    \item
    \hyperref[cl-gdata-service__class__gdata-error]{gdata-error}
    
	
    \end{itemize}
  
      


	
    \bigskip
    \textsc{Details}

Creates a new gdata-picasa object.


    
    \end{defun}
  
  
      \section{Other classes}
      

    \rule{\linewidth}{0.1mm}
    
    \label{cl-gdata-picasa__class__gdata-picasa}
    \begin{defun}[Class]
    gdata-picasa


      
    \bigskip
    \textsc{Superclasses}

\hyperref[cl-gdata-service__class__gdata-service]{
	  gdata-service
	}
      , \color[rgb]{0.5,0.5,0.5}common-lisp:standard-object\color[rgb]{0,0,0}, \color[rgb]{0.5,0.5,0.5}sb-pcl::slot-object\color[rgb]{0,0,0}, \color[rgb]{0.5,0.5,0.5}common-lisp:t\color[rgb]{0,0,0}


      
    \bigskip
    \textsc{Documented Subclasses}


	    None
	  


      
    \bigskip
    \textsc{Returned by}


 	
    \begin{itemize}
    
	  
    \item
    \hyperref[cl-gdata-picasa__fun__make-gdata-picasa]{make-gdata-picasa}
    
	
    \end{itemize}
  
      


      
    \bigskip
    \textsc{Inherited Slot Access Functions}


	
    \begin{itemize}
    
	  
    \item
    \hyperref[cl-gdata-service__fun__gdata-service-email]{gdata-service-email}
    
    \item
    \hyperref[cl-gdata-service__fun__gdata-service-password]{gdata-service-password}
    
    \item
    \hyperref[cl-gdata-service__fun__gdata-service-account-type]{gdata-service-account-type}
    
    \item
    \hyperref[cl-gdata-service__fun__gdata-service-service]{gdata-service-service}
    
    \item
    \hyperref[cl-gdata-service__fun__gdata-service-source]{gdata-service-source}
    
    \item
    \hyperref[cl-gdata-service__fun__gdata-service-auth]{gdata-service-auth}
    
	
    \end{itemize}
  
      


	
    \bigskip
    \textsc{Details}

A GData Picasa service.


    
    \end{defun}
  
  
      \section{Other variables}
      

    \rule{\linewidth}{0.1mm}
    
    \label{cl-gdata-picasa__variable___print-picasa_}
    \begin{defun}[Variable]
    *print-picasa*


	
    \bigskip
    \textsc{Details}

If T, write the customized presentation of the cl-gdata-picasa objects.


    
    \end{defun}
  
  

    \printindex
    \end{document}
  